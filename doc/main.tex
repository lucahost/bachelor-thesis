\documentclass{article}

\usepackage[ngerman]{babel}
\usepackage[a4paper,top=1.5cm,bottom=1.5cm,left=3cm,right=3cm,marginparwidth=1.75cm]{geometry}

\usepackage{graphicx}
\usepackage[table,xcdraw]{xcolor}

\usepackage[colorlinks=true, allcolors=blue]{hyperref}
\usepackage{amsmath,svg,amssymb,afterpage,tabto,parskip,minted,titling,fancyhdr,url,tcolorbox,lipsum}
\usepackage{subcaption}

\graphicspath{ {./images/} }

\definecolor{LightGray}{gray}{0.9}

\TabPositions{4cm}

\title{Strukturelle Untersuchungen der Genetischen Netzwerken von Alzheimer und Schizophrenie}
\author{Patrick Bernhardsgrütter \\ Luca Hostettler}
\date{January 2022}

\begin{document}
\begin{titlepage}
  \centering
  \includesvg[width=3cm]{logo_en.svg}\par\vspace{1cm}
  \linespread{1}\Large{\scshape Fernfachhochschule Schweiz\par}
  \vspace{1.5cm}
  {\scshape\Large\bfseries Semesterarbeit\par}
  {\huge\bfseries Strukturelle Untersuchungen der Genetischen Netzwerke von Alzheimer und Schizophrenie \par
    \Large\bfseries Network-Analysis\par}
  \vspace{1.5cm}
  \linespread{0.75}\large{Authoren:\par
    \linespread{0.75}\Large Patrick Bernhardsgrütter \\ Luca Hostettler }
  \vfill
  \linespread{0.75}\large{Eingereicht bei:\par
    Ao. Prof. Dr. habil. Matthias Dehmer}
  \vfill
  % Bottom of the page
  {\large \today \par}
\end{titlepage}

\newpage

%-------------------------------------------
%ABSTRACT
%-------------------------------------------

\pagenumbering{roman}
\section*{Abstract}

Der menschliche Organismus besteht aus einer Vielzahl an Netzwerken. Das Zusammenspiel von Genen in unseren Zellen wird heute noch aktiv erforscht. Es gibt Studien, welche sich dem Ziel annehmen, gewisse Ähnlichkeiten und Synergien genetischer Krankheiten zu erforschen.
Genetische Krankheiten werden in der Regel vererbt und befinden sich bereits bei der Geburt im Menschen.

Diese Arbeit entstand im Rahmen des Moduls \textbf{Network Analysis} der \textbf{Fernfachhochschule Schweiz} unter der Leitung von \textbf{Ao. Prof. Dr. habil. Dehmer}. Es werden in dieser Arbeit keine medizinischen Aussagen getroffen. Der zentrale Teil der Arbeit behandelt die Analyse von grossen Netzwerken und das Vergleichen von lokalen und globalen Messwerten der Netzwerke.

In dieser Arbeit werden Beziehungen zwischen den relevanten Genen von Alzheimer und Schizophrenie analysiert und miteinander verglichen.
Dabei wird zuerst eine Datenerhebung der relevanten und verwandten Gene der Krankheiten dokumentiert.

Als nächstes werden die Gene auf ihre Nachbarn, Gen-Fusionen oder gemeinschaftliche Experimente untersucht und daraus bilden sich die ersten Netzwerke.
Die Netzwerke werden anhand von verschiedenen Massen untersucht und miteinander verglichen.

Besonders bei der Bildung von Communities und den Zentralitätsmassen können wir bei den betroffenen Genen eine Ähnlichkeit feststellen.


\phantomsection

\newpage
\tableofcontents

\newpage

\pagestyle{fancy}
\fancyhf{}
\fancyhead[L]{Semesterarbeit Network-Analysis}
\fancyhead[R]{Patrick Bernhardsgrütter, Luca Hostettler}
\fancyfoot[L]{\thepage}
\fancyfoot[C]{\leftmark}
\fancyfoot[R]{\today}
\renewcommand{\headrulewidth}{2pt}
\renewcommand{\footrulewidth}{1pt}

\pagenumbering{arabic}

\input{chapters/01_introduction}
\input{chapters/02_baseline}
\input{chapters/03_theory}
\input{chapters/04_diseases}
\input{chapters/05_network_analysis}
\input{chapters/06_results}
\input{chapters/07_discussion}
\input{chapters/99_appendix}

\end{document}
